%\documentclass[aps,prb,twocolumn,groupedaddress,nofootinbib,floatfix]{revtex4}
\documentclass{article}
\usepackage{bm}
\usepackage{mathtools}
\usepackage{mathrsfs}
\usepackage[parfill]{parskip}

\usepackage[margin=1in]{geometry}

\usepackage{color}
\newcommand\editremark[1]{{\color{red}#1}}

\begin{document}

\title{EM PE Implementation and Usage}

%
\author{Ben Champion}

\date{\today}

\maketitle

\section{Models}

\subsection{Simple analytic kilonova model}

\textbf{Note:} some of this text comes directly from \cite{Villar_2017}.

The single-component, two-component, and three-component analytic kilonova models provided with the code are an implementation of the model described in \cite{Villar_2017}.
In the following equations, $M$ is the $r$-process ejecta mass and $v$ is the ejecta velocity.
Note that for now we assume the ejecta consists entirely of $r$-process material, so $M$ is the full ejecta mass.
The radioactive heating rate at time $t$ is given by \cite{Korobkin_2012}:
\begin{equation}
    L_\text{in}(t) = 4 \times 10^{18} M \times \left [ 0.5 - \pi^{-1} \arctan \left ( \frac {t - t_0} {\sigma} \right ) \right ]^{1.3} \text{ erg s}^{-1}
\end{equation}
where $t_0 = 1.3$ s and $\sigma = 0.11$ s are constants.

Only a fraction of $L_\text{in}$ powers the kilonova, given by the thermalization efficiency $\epsilon_\text{th}$.
This is approximated analytically in \cite{Barnes_2016}:
\begin{equation}
    \epsilon_\text{th}(t) = 0.36 \left [ e^{-a t} + \frac {\ln(1 + 2 b t^d)} {2 b t^d} \right ]
\end{equation}
The parameters $a$, $b$, and $d$ are constants that depend on the ejecta mass and velocity; an interpolation of Table 1 in \cite{Barnes_2016} is used in the model.

The bolometric luminosity is calculated as
\footnote{\cite{Villar_2017} is missing a factor of time in the denominator (this can be seen from the units -- our $L_\text{bol}$ has the correct units of erg s$^{-1}$). 
The factor of 1/$t_d$ seems to produce reasonable results with the correct units, and better aligns with e.g. \cite{Chatzopoulos_2012}.}
\begin{equation}
    L_\text{bol}(t) = \frac {2} {t_d} \exp{\left ( \frac {-t^2} {t_d^2} \right ) }
                    \int_0^t L_\text{in} \epsilon_\text{th} \exp{\left ( \frac {t^2} {t_d^2} \right ) } \frac {t} {t_d} dt
\end{equation}
where $t_d$ is the diffusion timescale, $t_d = \sqrt{2 \kappa M / \beta v c}$, $\kappa$ is the opacity, and $\beta = 13.7$ is a dimensionless constant related to the ejecta's geometry.

Lightcurves are calculated by assuming the kilonova behaves as a blackbody photosphere that expands at a velocity $v$.
The blackbody temperature is generally defined by its bolometric luminosity; however, once it cools to a critical temperature $T_c$, the photosphere recedes into the ejecta and the temperature remains fixed.
The photosphere temperature is
\footnote{\cite{Villar_2017} has a mistake here: $\sigma_\text{SB}$ should not be squared.}
\begin{equation} \label{T_c}
    T_\text{phot}(t) = \max \left [ \left ( \frac {L_\text{bol}(t)} {4 \pi \sigma_\text{SB} v^2 t^2} \right )^{1/4}, T_c \right]
\end{equation}

When $T_\text{phot} > T_c$, the photosphere radius is simply $R_\text{phot} = v t$.
When $T_\text{phot} = T_c$ (i.e. the photosphere has receded into the ejecta), the photosphere radius is
\begin{equation}
    R_\text{phot}(t) = \left ( \frac {L_\text{bol}(t)} {4 \pi \sigma_\text{SB} T_c^4} \right )^{1/2}
\end{equation}

The flux density at frequency $\nu$ is given in \cite{Metzger_2017}:
\begin{equation}
    F_\nu(t) = \frac {2 \pi h \nu^3} {c^2} \frac {1} {\exp{(h \nu / k T_\text{photo}(t))} - 1} \frac {R_\text{photo}^2(t)} {D^2}
\end{equation}
where $D$ is the source distance.
We use a fixed fiducial distance of $D = 10$ pc to calculate $F_\nu(t)$, then calculate AB magnitude with a distance modulus if necessary.

To compute multi-component lightcurves we assume each component has a photosphere that evolves independently of the others.
The total flux density is the sum of the flux densities of the individual components.
The version of the model implemented in the code uses three components with fixed opacities (a blue component with $\kappa = 0.5$ cm$^2$ g$^{-1}$, a purple component with $\kappa = 3$, cm$^2$ g$^{-1}$ and a red component with $\kappa = 10$ cm$^2$ g$^{-1}$).

\section{Parameter Estimation}

\subsection{Likelihood Function}

For lightcurve magnitudes $x_i$, model values $m_i(\bm{\theta})$ computed with a parameter vector $\bm{\theta}$, data uncertainties $\sigma_i$, and a scatter term $\sigma$, fit as a parameter, that accounts for additional uncertainty in the data and model, the log-likelihood is
\begin{equation}
    \ln \mathcal{L}(\bm{\theta}) = -0.5 \sum_{i=1}^n \left [ \frac {(x_i - m_i(\bm{\theta}))^2} {\sigma_i^2 + \sigma^2} + \ln(2 \pi (\sigma_i^2 + \sigma^2)) \right ]
\end{equation}
where the sum is taken over every data point in every band used in the analysis.

This log-likelihood is implemented as a method in \texttt{EM\_PE}'s \texttt{sampler} class, which can be imported and used in other codes.
The joint gravitational wave and EM log-likelihood is the sum of the individual log-likelihoods.

\subsection{Injection/Recovery Example}

A simple parameter estimation test using fake data can be run after installing the \texttt{EM\_PE} code to ensure it functions properly.
The repository contains a \texttt{Makefile} that generates reproducable parameter estimation runs.
To run the injection/recovery test:
\begin{verbatim}
$ make test_kilonova_3c
$ cd pe_runs/test_kilonova_3c/
$ ./sample.sh
\end{verbatim}
This series of commands will generate a posterior sample file, \texttt{samples.txt}.
Note that the default settings assume the code is being run on a machine with many CPU cores available, so it computes the likelihood in 8 parallel processes.
It will run without issue on a less powerful computer, but \texttt{sample.sh} should be modified to use fewer cores (e.g. set \texttt{--nprocs 2} to use two cores).

To produce a corner plot and lightcurve plot, run
\begin{verbatim}
$ ./plot_corner.sh
$ ./plot_lc.sh
\end{verbatim}
This will produce plots similar to Figure \ref{fig:corner} and Figure \ref{fig:lightcurves}.

\begin{figure}[!ht]
    \centering
    \includegraphics[width=6.5in]{corner.png}
    \caption{Posterior distribution for the injection/recovery test (the blue lines show the true parameter values). For this example, $T_c$ was fixed for each component and the distance was fixed to 40.0 Mpc.}
    \label{fig:corner}
\end{figure}

\begin{figure}[!ht]
    \centering
    \includegraphics[width=6.5in]{lc.png}
    \caption{Fake photometry data and lightcurve models for the injection/recovery test. The solid line shows the lightcurve model evaluated at the maximum-likelihood parameters.}
    \label{fig:lightcurves}
\end{figure}

\subsection{GW170817 Example Analysis}

\textit{Coming soon}

\clearpage

\bibliographystyle{h-physrev5.bst}
\bibliography{bibliography}

\end{document}

